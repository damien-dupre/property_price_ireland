\documentclass[]{elsarticle} %review=doublespace preprint=single 5p=2 column
%%% Begin My package additions %%%%%%%%%%%%%%%%%%%
\usepackage[hyphens]{url}

  \journal{An awesome journal} % Sets Journal name


\usepackage{lineno} % add
\providecommand{\tightlist}{%
  \setlength{\itemsep}{0pt}\setlength{\parskip}{0pt}}

\bibliographystyle{elsarticle-harv}
\biboptions{sort&compress} % For natbib
\usepackage{graphicx}
\usepackage{booktabs} % book-quality tables
%%%%%%%%%%%%%%%% end my additions to header

\usepackage[T1]{fontenc}
\usepackage{lmodern}
\usepackage{amssymb,amsmath}
\usepackage{ifxetex,ifluatex}
\usepackage{fixltx2e} % provides \textsubscript
% use upquote if available, for straight quotes in verbatim environments
\IfFileExists{upquote.sty}{\usepackage{upquote}}{}
\ifnum 0\ifxetex 1\fi\ifluatex 1\fi=0 % if pdftex
  \usepackage[utf8]{inputenc}
\else % if luatex or xelatex
  \usepackage{fontspec}
  \ifxetex
    \usepackage{xltxtra,xunicode}
  \fi
  \defaultfontfeatures{Mapping=tex-text,Scale=MatchLowercase}
  \newcommand{\euro}{€}
\fi
% use microtype if available
\IfFileExists{microtype.sty}{\usepackage{microtype}}{}
\ifxetex
  \usepackage[setpagesize=false, % page size defined by xetex
              unicode=false, % unicode breaks when used with xetex
              xetex]{hyperref}
\else
  \usepackage[unicode=true]{hyperref}
\fi
\hypersetup{breaklinks=true,
            bookmarks=true,
            pdfauthor={},
            pdftitle={Evaluation of property price fluctuaction according to geographical landmarks, an Dublin case study},
            colorlinks=false,
            urlcolor=blue,
            linkcolor=magenta,
            pdfborder={0 0 0}}
\urlstyle{same}  % don't use monospace font for urls

\setcounter{secnumdepth}{0}
% Pandoc toggle for numbering sections (defaults to be off)
\setcounter{secnumdepth}{0}
% Pandoc header
\usepackage{float}
\usepackage{booktabs}
\usepackage{array}
\usepackage{tabu}
\floatplacement{figure}{H}



\begin{document}
\begin{frontmatter}

  \title{Evaluation of property price fluctuaction according to geographical
landmarks, an Dublin case study}
    \author[Dublin City University]{Damien Dupré\corref{c1}}
   \ead{damien.dupre@dcu.ie} 
   \cortext[c1]{Corresponding Author}
      \address[Dublin City University]{Business School, Glasnevin, Dublin 9, Ireland}
  
  \begin{abstract}
  This is the abstract.
  \end{abstract}
  
 \end{frontmatter}

\section{Introduction}\label{introduction}

To acquire a property is one of the most important achievement that
individuals are seeking. It provides not only a housing security but
also the feeling of being a landowner. However the access to the status
of landowner is complicated because buying a property is the most
expensive spending of in a lifetime. For this reason understanding the
factors which are explaining how property prices evolve is a necessity.

Due to its geographic, economic and political situation, Ireland in
general and Dublin in particular saw important changes in property
prices in the last ten years. From a economic boom known as the ``Celtic
tiger'' in the 2000's, Ireland were deeply impacted by the 2007 economic
crisis. With an expected GDP growth of 4\% for 2019, property prices are
back to their highest. Whereas this grow is moderated in Irish mainland,
its capital Dublin is at the center of a housing crisis. Because of
factors including Irish economic wealth, the presence of tech companies
European headquarters such as Facebook or Google and the historic
configuration of the city which low population density structure and
underdeveloped public transportation, property prices became
unaffordable to most of Irish families.

In this paper we want to identify the spatio-temporal factors that
influenced the evolution of Dublin property prices. More precisely we
want to highlight not only macro economical influences such as GDP but
also the presence of economical landmarks such as tech companies
headquarters and public transportation system on property prices
evolution.

\section{Method}\label{method}

Since the 1st January 2010, under the Property Services (Regulation)
Act, all individuals acquiring a property in Ireland has to declare it
to Property Services Regulatory Authority (PSRA). It includes Date of
Sale, Price and Address of all residential properties purchased in
Ireland as declared to the Revenue Commissioners for stamp duty purposes
(https://propertypriceregister.ie). It must be noticed that data is
filed electronically by persons doing the conveyancing of the property
on behalf of the purchaser and errors may occur when the data is being
filed. In order to evaluate the spacial distribution of the property
sold, a geocoding from the filled addresses to GPS coordinated was
performed using the OpenStreetMap API.

\begin{table}[!h]

\caption{\label{tab:dublin-sample-size}Size of the PSRA database for properties sold in Dublin County per year since 2010 aftering filtering the orignal database.}
\centering
\fontsize{8}{10}\selectfont
\begin{tabular}{rr}
\toprule
year & n\\
\midrule
2010 & 4819\\
2011 & 3745\\
2012 & 4910\\
2013 & 5577\\
2014 & 832\\
2015 & 9657\\
2016 & 10731\\
2017 & 11991\\
2018 & 10395\\
\bottomrule
\end{tabular}
\end{table}

By focusing on the property sold in Dublin, 111155 entries were recorded
since 2010. After having filtered properties not corresponding to
houses, properties for which address was not possible to geocode,
artifacts in geocodes and aberrant value in sales price. From the
self-reported database, 62657 properties sold in Dublin between
2010-01-01 and 2018-11-30 was geocoded.

\end{document}


